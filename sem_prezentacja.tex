\input{preamble}
\usepackage{relsize}
\usepackage{lipsum}
\usepackage{alltt}

% c from texinfo.tex
\def\ifmonospace{\ifdim\fontdimen3\font=0pt }

% c C plus plus
\def\C++{%
  \ifmonospace%
  C++%
  \else%
  C\kern-.1667em\raise.30ex\hbox{\smaller{++}}%
  \fi%
  \spacefactor1000 }

% c C sharp
\def\Csharp{%
  \ifmonospace%
  C\#%
  \else%
  C\kern-.1667em\raise.30ex\hbox{\smaller{\#}}%
  \fi%
  \spacefactor1000 }


\begin{document}
% \usebackgroundtemplate{\includegraphics[width=\paperwidth]{img/bg.jpg}}
\frame{\titlepage}

% Przebieg prezentacji
\begin{frame}{Przebieg prezentacji}
  \tableofcontents
\end{frame}

% The following causes the table of contents to be shown 
% at the beginning of every subsection. Delete this, if you do not want it.
\AtBeginSubsection[]{
  \begin{frame}{Przebieg prezentacji}
    \tableofcontents[currentsection,currentsubsection]
  \end{frame}
}


\section{Wprowadzenie}

\subsection{Czym są inteligentne wskaźniki?}

\begin{frame}[t]{Idea inteligentnego wskaźnika}
  \begin{itemize}
  \item Inteligentne wskaźniki (ang. \emph{smart pointers}) są obiektami,
    które przechowują wskaźnik do dynamicznie zaalokowanych obiektów
    (przechowywanych na stercie). \pause
  \item Zachowują się bardzo podobnie do
    wskaźników wbudowanych w \C++ poza tym, że automatycznie usuwają wskazywany
    obiekt w odpowiednim momencie. \pause
  \item Prawdopodobnie najwięcej przypadków wadliwego działania
    programów napisanych w \C++ jest związanych z zarządzaniem
    pamięcią, wyciekami pamięci, błędami alokacji itp.
    Inteligentne wskaźniki wspomagają pisanie 
    programu odporne na takie błędy.
  \end{itemize}
\end{frame}





\subsection{Gdzie można znaleźć inteligentne wskaźniki?}
\begin{frame}{Biblioteka Boost}
  Biblioteka Boost w wersji 1.54 dostarcza 4 wskaźniki inteligentne:\pause
  \begin{itemize}
  	\item \texttt{scoped$\_$ptr}    \pause
  	\item \texttt{shared$\_$ptr}    \pause
  	\item \texttt{weak$\_$ptr}      \pause
  	\item \texttt{intrusive$\_$ptr} 
  \end{itemize}
  \pause
  Biblioteka udostępnia również tablice wskaźników inteligentnych:\pause
  \begin{itemize}
  \item	\texttt{scoped$\_$array} \pause
  \item	\texttt{shared$\_$array} 
  \end{itemize}
\end{frame}

\begin{frame}{Język  C++}
  Standard \C++11 wprowadził w nagłówku \texttt{<memory>}
  inteligentne wskaźniki na wzór biblioteki Boost: 
  \begin{itemize}
    \pause
  \item \texttt{std::unique$\_$ptr}, 
    \pause
  \item \texttt{std::shared$\_$ptr},
    \pause
  \item \texttt{std::weak$\_$ptr},
    \pause
  \item \texttt{std::bad$\_$weak$\_$ptr}
  \end{itemize}\pause
  Ich implementacja pojawiła się np. w środowisku 
  \textit{Visual Studio 2010}.	\pause
  Dokładny ich opis można znaleźć w normie
  \textit{ISO/IEC 14882:2011 (Information technology — Programming
    languages --- C++)} w rozdziale 20.7. 
\end{frame}


\section{Wskaźniki z języka \C++}

\subsection{\texttt{std::auto$\_$ptr} \tiny{(przestarzały, ang. depreciated)}}

\begin{frame}[t]{Opis działania}
  \footnotesize
  \only<1-5>{
    \begin{itemize}
    \item Szablon klasy \texttt{std::auto$\_$ptr} implementuje technikę RAII \\
      (ang. \textit{Resource Acquisition Is Initialization}) --- 
      inicjowanie przy pozyskaniu zasobu. \pause 
      
    \item Obiekt szablonu klasy \texttt{std::auto$\_$ptr} 
      jest inicjowany wskaźnikiem
      i można go używać jak wskaźnika. \pause
      
    \item Wskazywany obiekt będzie niejawnie usunięty 
      w chwili opuszczenia zasięgu
      obiektu szablonu klasy \texttt{auto$\_$ptr}.
      Szczególnie przydatne w połączeniu z obsługą wyjątków. 
      
      [\textbf{Przykład: }\textit{cpp$\_$auto$\_$ptr$\_$ExceptionSafe}]
      \pause
      
    \item Jeżeli więcej niż jeden \texttt{auto$\_$ptr} 
      wskazuje na ten sam obiekt, 
      to zachowanie jest niezdefiniowane (zazwyczaj 
      skończy się to błędem ochrony pamięci przy drugiej próbie zwolnienia zasobu). \pause
    \item Wzorzec klasy \texttt{auto\_ptr} dostarcza semantykę 
      ścisłej własności:
      po skopiowaniu jednego obiektu \texttt{auto$\_$ptr} na drugi, 
      obiekt źródłowy na nic już nie wskazuje 
      (przechowuje wskaźnik \texttt{NULL}).
      
      [\textbf{Przykład:} \textit{cpp$\_$auto$\_$ptr$\_$ScislaWlasnosc}]
  
		
    \end{itemize}
  }
  \only<6>{
    \begin{itemize}
      \item Konstruktor wzorca i przypisanie wzorca zapewniają, że 
        można dokonać niejawnej konwersji \texttt{std::auto$\_$ptr<D>} 
        do  \texttt{std::auto$\_$ptr<B>},
        jeśli można dokonać konwersji \texttt{D*} do \texttt{B*}
        (możliwe niejawne rzutowanie w górę hierarchii dziedziczenia) \pause
        
    \end{itemize}
  }

\end{frame}

\begin{frame}{Problemy związane z \texttt{std::auto$\_$ptr}}
  \begin{itemize}
  \small
  \item Po skopiowaniu jednego obiektu \texttt{std::auto$\_$ptr} na drugi,
    obiekt źródłowy ulega \textbf{zmodyfikowaniu} i na nic już nie wskazuje.

    Ponieważ kopiowanie takiego obiektu modyfikuje go, 
    nie można kopiować obiektów z atrybutem \texttt{const}
    (bardzo nieintuicyjne).
    
    [\textbf{przykład:} 
    \textit{cpp$\_$auto$\_$ptr$\_$kopiowanieStalegoObiektu}]
    
    \pause
    
    
          \item Nie da się stosować obiektów \texttt{auto$\_$ptr} jako elementów standardowych kolekcji (\texttt{std::list}, \texttt{std::vector}, itp.). Kolekcje STL nakładają wymagania na swoje elementy, których \texttt{auto$\_$ptr}
          nie spełnia.
      Nie powinno się stosować jako typ elementów tablic --- próba posortowania 
      takiej tablicy może przynieść złe efekty (niezgodny ze standardową biblioteką algorytmów). \pause
      
      \item Wzorzec \texttt{std::auto$\_$ptr} wyszedł z użytku (jest niezalecany). Mimo wszystko często można go spotkać w projektach. De facto
      nie jest to wskaźnik inteligentny. Standard \C++11 wprowadził \texttt{std::unique$\_$ptr}, który znacznie lepiej radzi sobie od jego kulawego poprzednika.
  \end{itemize}
\end{frame}

\subsection{\texttt{std::unique$\_$ptr} \tiny{(\C++11)}}

\begin{frame}[t]{Opis działania}
  \footnotesize
  \only<1-7>{
  \begin{itemize}
  	\item Przechowuje wskaźnik do obiektu zaalokowanego dynamicznie.
  	  \pause
  	
  	\item Tak jak \texttt{std::auto$\_$ptr} dostarcza semantykę ścisłej własności.
  	  Dwie instancje \texttt{std::unique$\_$ptr} nie mogą wskazywać tego samego obiektu
  	  jednocześnie.
  	  \pause
  	  
 	    	  
  	\item Można dostarczyć do niego odpowiednią implementację (tzw. deleter), która
  	  zajmie się usuwaniem zarządzanego obiektu.
  	  \pause
  	  
  	\item Bezpieczniejszy niż \texttt{std::auto$\_$ptr}, współpracuje z kolejkami STL.
  	  \pause
  	  
  	\item Do przeniesienia odpowiedzialności na inny obiekt
  	  \texttt{std::unique$\_$ptr} za wskazywany obiekt wykorzystywana jest
  	  funkcja \texttt{std::move}. 
  	  
  	  [\textbf{Przykład: }\textit{cpp$\_$unique$\_$ptr$\_$move}]
  	  \pause
  	  
  	\item Do zwolnienia odpowiedzialności za wskazywany obiekt służy
  	  metoda \texttt{std::unique$\_$ptr::release}.
  	  
  	  [\textbf{Przykład: }\textit{cpp$\_$unique$\_$ptr$\_$release}]
  	  \pause
  	
  	\item W celu usunięcia wskazywanego obiektu i zaalokowania nowego służy metoda
  	  \texttt{std::unique$\_$ptr::reset}
  \end{itemize}
  }
  \only<8->{
  	\begin{itemize}
  	  \item Do zamiany odpowiedzialności za wskazywane obiekty między 
  	    obiektami \texttt{unique$\_$ptr}
  	    służy metoda \texttt{std::unique$\_$ptr::swap}
  	    	  
  	    [\textbf{Przykład: }\textit{cpp$\_$unique$\_$ptr$\_$swap}]
  	    \pause
  	    	  
  	\end{itemize}
  }
\end{frame}

\section{Inteligentne wskaźniki z biblioteki Boost}

\subsection{\texttt{scoped$\_$ptr}}

\begin{frame}[t]{Opis działania}
  \footnotesize
  
  \only<1-5>{
  \begin{itemize}
    \item Dostarcza funkcjonalność RAII (inicjowanie przy pozyskaniu zasobu).
      Nie dostarcza mechanizmów wspólnej własności i przekazywania własności. 
      \pause
    
    \item Nazwa wzorca oraz wprowadzona semantyka (nie da się kopiować obiektu tej klasy)
      sygnalizuje, że obiekt wzorca \texttt{scoped$\_$pointer} zamierza zachować 
      prawo własności wyłącznie w bieżącym zakresie. W trakcie 
      wyjścia z zakresu mamy gwarancję usunięcia zarządzanego obiektu.
      
      $[$\textbf{Przykład: }\textit{cpp$\_$unique$\_$ptr$\_$swap}$]$
      \pause
      
    \item Z uwagi na brak możliwości kopiowania inteligentnych wskaźników 
      \texttt{scoped$\_$pointer}, jest bezpieczniejszy od wskaźników
      \texttt{shared$\_$ptr} lub \texttt{std::auto$\_$ptr} dla wskaźników,
      które nie powinny być kopiowane.
      \pause      
      
    \item Ponieważ \texttt{scoped$\_$ptr} posiada prostą implementację,
      każda operacja jest tak szybka jak dla wbudowanego wskaźnika
      i nie ma większych narzutów pamięciowych niż wbudowany wskaźnik.
      \pause
      
    \item \texttt{scoped$\_$ptr} nie może być używany w połączeniu z biblioteką 
      standardową C++.
      Należy użyć \texttt{shared$\_$ptr}, jeżeli potrzebujemy inteligentnego wskaźnika,
      który to umożliwia.
  \end{itemize}
  }
  \only<6>
  {
    \begin{itemize}
      \item \texttt{scoped$\_$ptr} nie może poprawnie zarządzać 
        dynamicznie utworzoną tablicą obiektów --- w tym celu należy użyć 
        \texttt{scoped$\_$array}.\pause
        
        
    \end{itemize}
  }
\end{frame}

\subsection{\texttt{shared$\_$ptr}}
\begin{frame}[t]{Opis działania}
  \footnotesize
  \only<1-5>
  {
    \begin{itemize}
      \item Szablon klasy \texttt{shared$\_$ptr} przechowuje wskaźnik do dynamicznie
        zaalokowanego obiektu (przeważnie za pomocą operatora \texttt{new}).
        \pause
        
      \item Wskazywany obiekt jest niszczony, kiedy ostatni wskazujący go obiekt      
        \texttt{shared$\_$ptr} zostanie zniszczony lub wywołana zostanie jego metoda 
        \texttt{reset}.
        \pause
      
      \item Każdy \texttt{shared$\_$ptr} spełnia wymogi standardowej biblioteki \C++
        i może być z nią używany (\texttt{CopyConstructible}, \texttt{MoveConstructible},
        \texttt{MoveAssignable}).  
        \pause
      
      \item \texttt{shared$\_$ptr} usuwa dokładnie taki wskaźnik, jaki został podany w
        momencie konstrukcji. Np:
      
        \texttt{shared$\_$ptr<void> p1( new int(5) );}
      
        usuwanie w taki sposób zaalokowanego obiektu wiąże się z wywołaniem operatora
        \texttt{\textbf{delete}} na wskaźniku typu \texttt{\textbf{int*}} mimo,
        że \texttt{p1} jest typu \texttt{shared$\_$ptr<void> }
        \pause
      
      \item Szablon klasy parametryzowany jest literą \texttt{T}, typem obiektu
        wskazywanego przez obiekt. \texttt{T} może być typem niekompletnym lub
        typu \texttt{void}. Szablon jak i większość jego metod nie nakłada wymagań na 
        wskazywany typ \texttt{T}.
    \end{itemize}
  }
  \only<6->{
  	\begin{itemize}
  	  \item \texttt{shared$\_$ptr<T>} może być niejawnie przekonwertowany do 
  	    \texttt{shared$\_$ptr<U>} jeżeli \texttt{T*} może być niejawnie
  	    przekonwertowane do \texttt{U*}.
  	    
  	  \item W bibliotece Boost począwszy od wersji 1.53, \texttt{shared$\_$ptr}
  	    może być użyty do przechowywania wskaźnika do dynamicznie zaalokowanej
  	    tablicy.
  	\end{itemize}
  }
\end{frame}

\begin{frame}[t]{Dobre nawyki}
\footnotesize

\begin{itemize}
	\item Prosta wskazówka, która prawie całkowicie eliminuje możliwość
	  wycieków pamięci: używać nazwanego wskaźnika inteligentnego do 
	  przechowywania rezultatu wywołania operatora \texttt{new}.
	  
	\item PRZENIGDY nie używać nienazwanych inteligentnych wskaźników
	jako obiekty tymczasowe.
	
	$[$\textbf{Przykład: }\textit{cpp$\_$shared$\_$ptr$\_$unnamed}$]$
\end{itemize}

\end{frame}

\subsection{\texttt{weak$\_$ptr}}
\begin{frame}[t]{Opis działania}
	\footnotesize
	
	\only<1->
	{
	  \begin{itemize}
	    \item Szablon klasy \texttt{weak$\_$ptr} przechowuje ,,słabe odwołanie''
	      do obiektu, który posiada już odwołanie w \texttt{shared$\_$ptr}.
	      \pause
	      
	    \item Aby móc  dostać się do wskazywanego obiektu,
	      \texttt{weak$\_$ptr} może zostać
	      skonwertowany do \texttt{shared\_ptr} za pomocą konstruktora 
	      \texttt{shared$\_$ptr} lub za pomocą metody \texttt{lock}
	      \pause
	    \item 
	  \end{itemize}
	}
\end{frame}

\subsection{\texttt{intrusive$\_$ptr}}

\section{}
\begin{frame}
  \begin{center}
    \huge
    Dziękuję za uwagę!
  \end{center}
 
  
\end{frame}

\end{document}
