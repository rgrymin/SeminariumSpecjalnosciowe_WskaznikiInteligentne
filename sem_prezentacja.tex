\input{preamble}
\usepackage{relsize}
\usepackage{lipsum}
\usepackage{alltt}

% c from texinfo.tex
\def\ifmonospace{\ifdim\fontdimen3\font=0pt }

% c C plus plus
\def\C++{%
  \ifmonospace%
  C++%
  \else%
  C\kern-.1667em\raise.30ex\hbox{\smaller{++}}%
  \fi%
  \spacefactor1000 }

% c C sharp
\def\Csharp{%
  \ifmonospace%
  C\#%
  \else%
  C\kern-.1667em\raise.30ex\hbox{\smaller{\#}}%
  \fi%
  \spacefactor1000 }


\begin{document}
% \usebackgroundtemplate{\includegraphics[width=\paperwidth]{img/bg.jpg}}
\frame{\titlepage}

% Przebieg prezentacji
\begin{frame}{Przebieg prezentacji}
  \tableofcontents
\end{frame}

% The following causes the table of contents to be shown 
% at the beginning of every subsection. Delete this, if you do not want it.
\AtBeginSubsection[]{
  \begin{frame}{Przebieg prezentacji}
    \tableofcontents[currentsection,currentsubsection]
  \end{frame}
}


\section{Wprowadzenie}

\subsection{Czym są inteligentne wskaźniki?}

\begin{frame}{Idea inteligentnego wskaźnika}
  \begin{itemize}
  \item Inteligentne wskaźniki (ang. \emph{smart pointers}) są obiektami,
    które przechowują wskaźnik do dynamicznie zaalokowanych obiektów
    (przechowywanych na stercie). \pause
  \item Zachowują się bardzo podobnie do
    wskaźników wbudowanych w \C++ poza tym, że automatycznie usuwają wskazywany
    obiekt w odpowiednim momencie. \pause
  \item Prawdopodobnie najwięcej przypadków wadliwego działania
    programów napisanych w \C++ jest związanych z zarządzaniem
    pamięcią, wyciekami pamięci, błędami alokacji itp.
    Inteligentne wskaźniki wspomagają pisanie 
    programu odporne na takie błędy.
  \end{itemize}
\end{frame}


\subsection{Korzyści płynące z wykorzystania inteligentnych wskaźników}

\begin{frame}{Mniej błędów}
\end{frame}

\begin{frame}{,,Exception safety''}
\end{frame}

\begin{frame}{Odśmiecacz (ang. \textit{garbage collector})}
  \C++ nie dostarcza mechanizmu odśmiecania (mechanizmu, który w osobnym wątku
  nadzoruje użyteczność przydzielonych zasobów i zwalnia je, 
  kiedy są niepotrzebne). \pause
  
  Można napisać własny mechanizm odśmiecania, który będzie współpracował
  z odpowiednio napisanymi inteligentnymi wskaźnikami.
\end{frame}

\begin{frame}{Kontenery STL}  
  \begin{minipage}[t][.75\textheight]{\textwidth}
    \begin{block}{Problem}
      Co powinniśmy zrobić, jeżeli potrzebujemy
    \end{block}	
  \end{minipage}
\end{frame}


\subsection{Gdzie można znaleźć inteligentne wskaźniki?}
\begin{frame}{Boost}
  
\end{frame}

\begin{frame}{Język  C++}
  Standard \C++11 wprowadził w nagłówku \texttt{<memory>}
  inteligentne wskaźniki na wzór biblioteki Boost: 
  \begin{itemize}
    \pause
  \item \texttt{std::unique$\_$ptr}, 
    \pause
  \item \texttt{std::shared$\_$ptr},
    \pause
  \item \texttt{std::weak$\_$ptr},
    \pause
  \item \texttt{std::bad$\_$weak$\_$ptr}
  \end{itemize}\pause
  Ich implementacja pojawiła się np. w środowisku 
  \textit{Visual Studio 2010}.	\pause
  Dokładny ich opis można znaleźć w normie
  \textit{ISO/IEC 14882:2011 (Information technology — Programming
    languages --- C++)} w rozdziale 20.7. 
\end{frame}


\section{Wskaźniki z języka \C++}

\subsection{\texttt{std::auto$\_$ptr} \tiny{(przestarzały, ang. deprecated)}}

\begin{frame}{Opis działania}
  % \begin{minipage}[t][\textheight]{\textwidth}
  \footnotesize
  \only<1-6>{
    \begin{itemize}
    \item Szablon klasy \texttt{std::auto$\_$ptr} implementuje technikę RAII \\
      (ang. \textit{Resource Acquisition Is Initialization}) --- 
      inicjowanie przy pozyskaniu zasobu. \pause 
      
    \item Obiekt szablonu klasy \texttt{std::auto$\_$ptr} 
      jest inicjowany wskaźnikiem
      i można go używać jak wskaźnika. \pause
      
    \item Wskazywany obiekt będzie niejawnie usunięty 
      w chwili opuszczenia zasięgu
      obiektu szablonu klasy \texttt{auto$\_$ptr}. \pause
      
    \item Jeżeli więcej niż jeden \texttt{auto$\_$ptr} 
      wskazuje na ten sam obiekt, 
      to zachowanie jest niezdefiniowane (zazwyczaj 
      skończy się to błędem ochrony pamięci przy drugiej próbie zwolnienia zasobu). \pause
    \item Wzorzec klasy \texttt{auto\_ptr} dostarcza semantykę 
      ścisłej własności:
      po skopiowaniu jednego obiektu \texttt{auto$\_$ptr} na drugi, 
      obiekt źródłowy na nic już nie wskazuje 
      (przechowuje wskaźnik \texttt{NULL}).	\pause
		
    \item Konstruktor wzorca i przypisanie wzorca zapewniają, że 
      można dokonać niejawnej konwersji \texttt{std::auto$\_$ptr<D>} 
      do  \texttt{std::auto$\_$ptr<B>},
      jeśli można dokonać konwersji \texttt{D*} do \texttt{B*}.
    \end{itemize}
  }
  \only<7>{
    \begin{center}
      \Large
      \vspace{0.5in}
      \textbf{Przykłady:}\\
      \textit{cpp$\_$auto$\_$ptr$\_$ScislaWlasnosc}\\
      \textit{cpp$\_$auto$\_$ptr$\_$ThreadSafe}
    \end{center}
  }
  % \end{minipage}
\end{frame}

\begin{frame}{Problemy związane z \texttt{std::auto$\_$ptr}}
  \begin{itemize}
  \item Po skopiowaniu jednego obiektu \texttt{auto$\_$ptr} na drugi,
    obiekt źródłowy ulega \textbf{zmodyfikowaniu} i na nic już nie wskazuje.
    \pause
    Ponieważ kopiowanie takiego obiektu modyfikuje go, 
    nie można kopiować obiektów z atrybutem \texttt{const}
    (bardzo nieintuicyjne).
    [\textbf{przykład:} 
    \textit{cpp$\_$auto$\_$ptr$\_$kopiowanieStalegoObiektu}]
  \end{itemize}
\end{frame}

\subsection{\texttt{std::unique$\_$ptr} \tiny{(\C++11)}}


\section{Inteligentne wskaźniki z biblioteki Boost}
\subsection{\texttt{scoped$\_$ptr}}

\subsection{\texttt{shared$\_$ptr}}

\subsection{\texttt{weak$\_$ptr}}

\subsection{\texttt{intrusive$\_$ptr}}

\section{}
\begin{frame}
  \begin{center}
    \huge
    Dziękuję za uwagę!
  \end{center}
 
  
\end{frame}

\end{document}
